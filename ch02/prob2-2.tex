\documentclass[11pt]{article}          % use "amsart" instead of "article" for AMSLaTeX format
\pagestyle{empty}
\usepackage[margin=0.5in]{geometry}                           % See geometry.pdf to learn the layout options. There are lots.
\geometry{a4paper}
%\geometry{landscape}                           % Activate for for rotated page geometry
\usepackage[parfill]{parskip}                   % Activate to begin paragraphs with an empty line rather than an indent
\usepackage{graphicx}                           % Use pdf, png, jpg, or eps§ with pdflatex; use eps in DVI mode
                                                                % TeX will automatically convert eps --> pdf in pdflatex                
\usepackage{amsmath}
\usepackage{amssymb}
\usepackage{enumitem}
\usepackage{mathtools}
\usepackage{clrscode3e}

\DeclarePairedDelimiter{\floor}{\lfloor}{\rfloor}

\begin{document}
%\section{}
%\subsection{}

\textbf{2-2 Correctness of bubblesort}

\textbf{a.} In order to show that BUBBLESORT actually sorts, what else do we need to prove?

\textbf{Answer}

We need to prove that the initialization holds the invariant of BUBBLESORT.

\textbf{b.} 

\textbf{Answer}

The loop invariant is always keep the larger value to the right.

\textbf{Initialization} We start proving that the invariant holds before the first iteration
with $j = n$. Since, the $A[n+1]$ does not exist, we may conclude that $A[n]$ is smaller than
its next value.

\textbf{Maintenance} For each iteration, we swaps the larger element to one position right,
so we left with $A'[j-1] <= A'[j]$ holding the invariant of keeping the larger value to the
right.

\textbf{Termination} We end with $j = i + 1$, since we swaps the current element with the
previous one, we always endup with the smallest element on $A[i]$.

\textbf{c.} 

\textbf{Answer}

The loop invariant is always keep the smallest elements to left

\textbf{Initialization} 

\textbf{Maintenance} 

\textbf{Termination} We finish with the subarray $A[i..n-1]$ elements sorted. Since the left
subarray always holds the largest element, we don't need to sort the $A[n]$.

\textbf{d. What is the worst-case running time of bubblesort? How does it compare to the running
time of insertion sort?}

\textbf{Answer}

The $T(n) = (n - 1) + (n - 2) + ... + 1 = \frac{n(n - 1)}{2}$, so BUBBLESORT has $\Theta (n^2)$ the
same as INSERTION-SORT.

\end{document}
