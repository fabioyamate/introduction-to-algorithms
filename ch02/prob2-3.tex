\documentclass[11pt]{article}          % use "amsart" instead of "article" for AMSLaTeX format
\pagestyle{empty}
\usepackage[margin=0.5in]{geometry}                           % See geometry.pdf to learn the layout options. There are lots.
\geometry{a4paper}
%\geometry{landscape}                           % Activate for for rotated page geometry
\usepackage[parfill]{parskip}                   % Activate to begin paragraphs with an empty line rather than an indent
\usepackage{graphicx}                           % Use pdf, png, jpg, or eps§ with pdflatex; use eps in DVI mode
                                                                % TeX will automatically convert eps --> pdf in pdflatex                
\usepackage{amsmath}
\usepackage{amssymb}
\usepackage{enumitem}
\usepackage{mathtools}
\usepackage{clrscode3e}

\DeclarePairedDelimiter{\floor}{\lfloor}{\rfloor}

\begin{document}
%\section{}
%\subsection{}

\textbf{2-3 Correctness of Horner’s rule}

\textbf{a.} In terms of‚ $\Theta$-notation, what is the running time of this code fragment for Horner’s rule?

\textbf{Answer}

$\Theta (n)$

\textbf{b.} Write pseudocode to implement the naive polynomial-evaluation algorithm that computes
each term of the polynomial from scratch. What is the running time of this algorithm? How does it
compare to Horner’s rule?

\textbf{Answer}

\begin{codebox}
\Procname{$\proc{NAIVE-POLINOMIAL-EVALUATION}(A, x)$}
\li $\id{sum} \gets 0$
\li \For $j \gets 1$ \To $\attrib{A}{length} - 1$
    \Do
\li       $y \gets A[j]$
\li       \For $i \gets 1$ \To $j$
          \Do
\li             $y \gets y \cdot x$
          \End
\li       $\id{sum} \gets \id{sum} + y$
    \End
\end{codebox}

This naive implementation has $T(n) = n (n - 1) (n - 2) ... (n - k) = n!$. So, the worst-case
running time is $\Theta (n!)$. As we seem on \textbf{prob1-1} the $n!$ is pretty bad compared
to the linear complexity of Horner's rule.

\textbf{c.}

\textbf{Answer}

\textbf{d.}

\textbf{Answer}

\end{document}
